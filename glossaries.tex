%Glossary

\newglossaryentry{word}{name=Microsoft Word, description={is a word processor published by Microsoft. It is one of the office productivity applications included in the Microsoft Office suite.\cite{wikipedia}}}

\newglossaryentry{pp}{name=Microsoft PowerPoint, description={is a presentation program, published by Microsoft. It is one of the office productivity applications included in the Microsoft Office suite.\cite{wikipedia}}}

\newglossaryentry{teams}{name=Microsoft Teams, description={is a business communication platform, published by Microsoft. It is one of the office productivity applications included in the Microsoft Office suite.\cite{wikipedia}}}

\newglossaryentry{drive}{name=Microsoft OneDrive, description={is a file hosting service and synchronization service, published by Microsoft. It is one of the office productivity applications included in the Microsoft Office suite.\cite{wikipedia}}}

\newglossaryentry{outlook}{name=Microsoft Outlook, description={is s a personal information manager web app from Microsoft consisting of webmail, calendaring, contacts, and tasks services. It is one of the office productivity applications included in the Microsoft Office suite.\cite{wikipedia}}}

\newglossaryentry{office}{name=Microsoft Office, description={is a family of client software, server software, and services developed by Microsoft.\cite{wikipedia}}}

\newglossaryentry{kb}{name=Knowledge Base, description={a place for creating, sharing, using and managing the knowledge and information of an organization.\cite{wikipedia}}}

\newglossaryentry{anonymous}{name=Anonymous, description={is a decentralized international activist/hacktivist collective/movement that is widely known for its various cyber attacks against several governments, government institutions and government agencies, corporations, and the Church of Scientology.\cite{wikipedia}}}

\newglossaryentry{repository}{name=Repository, description={,in revision control systems, is a data structure that stores metadata for a set of files or directory structure.\cite{wikipedia}}}

%Glossary Acronyms entry

\newglossaryentry{slag}{name=Service-level Agreement, description={is a commitment between a service provider and a client. Particular aspects of the service {–} quality, availability, responsibilities {–} are agreed between the service provider and the service user.\cite{wikipedia}}}

\newglossaryentry{pstng}{name=Public switched telephone network, description={the aggregate of the world's circuit-switched telephone networks that are operated by national, regional, or local telephony operators, providing infrastructure and services for public telecommunication.\cite{wikipedia}}}

\newglossaryentry{cccg}{type=\glsdefaulttype, name=Chaos Computer Club, description={is Europe's largest association of hackers with 7700 registered members.\cite{wikipedia}}}

\newglossaryentry{iacg}{type=\glsdefaulttype, name=Infrastructure as Code, description={is the process of managing and provisioning computer data centers through machine-readable definition files, rather than physical hardware configuration or interactive configuration tools.\cite{wikipedia}}}

\newglossaryentry{hag}{type=\glsdefaulttype, name=High Availability, description={is a characteristic of a system which aims to ensure an agreed level of operational performance, usually uptime, for a higher than normal period.\cite{wikipedia}}}

\newglossaryentry{mdmg}{type=\glsdefaulttype, name=Mobile device management, description={is typically a deployment of a combination of on-device applications and configurations, corporate policies and certificates, and backend infrastructure, for the purpose of simplifying and enhancing the IT management of end user devices.\cite{wikipedia}}}

\newglossaryentry{adg}{type=\glsdefaulttype, name=Active Directory, description={ is a directory service developed by Microsoft for Windows domain networks. It is included in most Windows Server operating systems as a set of processes and services. Initially, Active Directory was used only for centralized domain management. However, Active Directory eventually became an umbrella title for a broad range of directory-based identity-related services.\cite{wikipedia}}}

\newglossaryentry{polpg}{name=Principle of least privilege, description={ also known as the principle of minimal privilege or the principle of least authority, requires that in a particular abstraction layer of a computing environment, every module (such as a process, a user, or a program, depending on the subject) must be able to access only the information and resources that are necessary for its legitimate purpose.\cite{wikipedia}}}

\newglossaryentry{itilg}{name=Information Technology Infrastructure Library, description={is a set of detailed practices for IT activities that describes processes, procedures, tasks, and checklists which are neither organization-specific nor technology-specific, but can be applied by an organization toward strategy, delivering value, and maintaining a minimum level of competency.\cite{wikipedia}}}

\newglossaryentry{mttrg}{name=Mean time to repair, description={is a basic measure of the maintainability of repairable items. It represents the average time required to repair a failed component or device. Expressed mathematically, it is the total corrective maintenance time for failures divided by the total number of corrective maintenance actions for failures during a given period of time.\cite{wikipedia}}}

\newglossaryentry{iscedg}{name=International Standard Classification of Education, description={is a statistical framework for organizing information on education maintained by the \textit{\gls{unesco}}. It is a member of the international family of economic and social classifications of the United Nations.\cite{wikipedia}}}

\newglossaryentry{cefrg}{name=Common European Framework of Reference for Languages, description={is a guideline used to describe achievements of learners of foreign languages across Europe and, increasingly, in other countries. The CEFR is also intended to make it easier for educational institutions and employers to evaluate the language qualifications of candidates to education admission or employment.\cite{wikipedia}}}

%Acronyms

\newglossaryentry{perc}{type=\acronymtype, name={PERC}, description={The Propane Education \& Research Council}}

\newglossaryentry{smart}{type=\acronymtype, name={S.M.A.R.T.}, description={Specific, Measurable, Achievable, Relevant and Time-based}}

\newglossaryentry{etz}{type=\acronymtype, name={ETZ}, description={Eastern Time Zone}}

\newglossaryentry{st}{type=\acronymtype, name={S\&T}, description={Sales and Training}}

\newglossaryentry{api}{type=\acronymtype, name={API}, description={Application Programming Interface}}

\newglossaryentry{fifo}{type=\acronymtype, name={FIFO}, description={First In, First Out}}

\newglossaryentry{raci}{type=\acronymtype, name={RACI}, description={Responsible Accountable Consulted Informed}}

\newglossaryentry{vm}{type=\acronymtype, name={VM}, description={Virtual Machine}}

\newglossaryentry{msp}{type=\acronymtype, name={MSP}, description={Managed Service Provider}}

\newglossaryentry{unesco}{type=\acronymtype, name={UNESCO}, description={United Nations Educational, Scientific and Cultural Organization}}

\newglossaryentry{rfp}{type=\acronymtype, name={RFP}, description={Request for Proposal}}

\newglossaryentry{pm}{type=\acronymtype, name={PM}, description={Post meridiem}}

\newglossaryentry{am}{type=\acronymtype, name={AM}, description={Ante meridiem}}

%Acronyms glossary entry

\newglossaryentry{sla}{type=\acronymtype, name={SLA}, description={Service-Level Agreement}, first={SLA\glsadd{slag}}, see=[Glossary:]{slag}}

\newglossaryentry{pstn}{type=\acronymtype, name={PSTN}, description={Public switched telephone network}, first={PSTN\glsadd{pstng}}, see=[Glossary:]{pstng}}

\newglossaryentry{ccc}{type=\acronymtype, name={CCC}, description={Chaos Computer Club}, first={CCC\glsadd{cccg}}, see=[Glossary:]{cccg}}

\newglossaryentry{iac}{type=\acronymtype, name={IaC}, description={Infrastructure as Code}, first={IaC\glsadd{iacg}}, see=[Glossary:]{iacg}}

\newglossaryentry{ha}{type=\acronymtype, name={HA}, description={High Availability}, first={HA\glsadd{hag}}, see=[Glossary:]{hag}}

\newglossaryentry{ad}{type=\acronymtype, name={AD}, description={Active Directory}, first={AD\glsadd{adg}}, see=[Glossary:]{adg}}

\newglossaryentry{mdm}{type=\acronymtype, name={MDM}, description={Mobile device management}, first={MDM\glsadd{mdmg}}, see=[Glossary:]{mdmg}}

\newglossaryentry{polp}{type=\acronymtype, name={PoLP}, description={Principle of least privilege}, first={PoLP\glsadd{polpg}}, see=[Glossary:]{polpg}}

\newglossaryentry{itil}{type=\acronymtype, name={ITIL}, description={Information Technology Infrastructure Library}, first={ITIL\glsadd{itilg}}, see=[Glossary:]{itilg}}

\newglossaryentry{mttr}{type=\acronymtype, name={MTTR}, description={Mean time to repair}, first={MTTR\glsadd{mttrg}}, see=[Glossary:]{mttrg}}

\newglossaryentry{isced}{type=\acronymtype, name={ISCED}, description={United Nations Educational, Scientific and Cultural Organization}, first={ISCED\glsadd{iscedg}}, see=[Glossary:]{iscedg}}

\newglossaryentry{cefr}{type=\acronymtype, name={CEFR}, description={Common European Framework of Reference for Languages}, first={CEFR\glsadd{cefrg}}}

